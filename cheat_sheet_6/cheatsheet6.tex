\documentclass[10pt,landscape]{article}
\usepackage{multicol}
\usepackage{calc}
\usepackage{ifthen}
\usepackage[landscape]{geometry}
\usepackage{hyperref}
\usepackage{cite}


% To Do:
% \listoffigures \listoftables
% \setcounter{secnumdepth}{0}


% This sets page margins to .5 inch if using letter paper, and to 1cm
% if using A4 paper. (This probably isn't strictly necessary.)
% If using another size paper, use default 1cm margins.
\ifthenelse{\lengthtest { \paperwidth = 11in}}
	{ \geometry{top=.5in,left=.5in,right=.5in,bottom=.5in} }
	{\ifthenelse{ \lengthtest{ \paperwidth = 297mm}}
		{\geometry{top=1cm,left=1cm,right=1cm,bottom=1cm} }
		{\geometry{top=1cm,left=1cm,right=1cm,bottom=1cm} }
	}

% Turn off header and footer
\pagestyle{empty}
 

% Redefine section commands to use less space
\makeatletter
\renewcommand{\section}{\@startsection{section}{1}{0mm}%
                                {-1ex plus -.5ex minus -.2ex}%
                                {0.5ex plus .2ex}%x
                                {\normalfont\large\bfseries}}
\renewcommand{\subsection}{\@startsection{subsection}{2}{0mm}%
                                {-1explus -.5ex minus -.2ex}%
                                {0.5ex plus .2ex}%
                                {\normalfont\normalsize\bfseries}}
\renewcommand{\subsubsection}{\@startsection{subsubsection}{3}{0mm}%
                                {-1ex plus -.5ex minus -.2ex}%
                                {1ex plus .2ex}%
                                {\normalfont\small\bfseries}}
\makeatother

% Define BibTeX command
\def\BibTeX{{\rm B\kern-.05em{\sc i\kern-.025em b}\kern-.08em
    T\kern-.1667em\lower.7ex\hbox{E}\kern-.125emX}}

% Don't print section numbers
\setcounter{secnumdepth}{0}


\setlength{\parindent}{0pt}
\setlength{\parskip}{0pt plus 0.5ex}


% -----------------------------------------------------------------------

\begin{document}

\raggedright
\footnotesize
\begin{multicols}{3}


% multicol parameters
% These lengths are set only within the two main columns
%\setlength{\columnseprule}{0.25pt}
\setlength{\premulticols}{1pt}
\setlength{\postmulticols}{1pt}
\setlength{\multicolsep}{1pt}
\setlength{\columnsep}{2pt}

\begin{center}
     \Large{\textbf{JAVA Cheat Sheet 6}} \\
\end{center}

\section{I/O}
"A \textbf{stream} is a sequence if data."\\
"A program uses an \textbf{input stream} to read data from a source."\\
"A program uses an \textbf{output stream} to write data to a destination."  \textit{(docs.oracle.com)}\\
\subsection{Byte Streams}
In Java all stream types are built on \textbf{ByteStreams}.\\
\textit{FileInputStream in = new FileInputStream(fileName);\\
FileOutputStream out = new FileOutputStream(fileName);\\
int c;\\
while((c = in.read()) != -1)\{\\
\hspace{2mm} out.write(c);\\
\}\\}
\subsection{Character Streams}
\textbf{FileReader} and \textbf{FileWriter} use \textit{FileInputStream} and \textit{FileOutputStream} internally.\\
\textit{FileReader in = new FileReader(fileName);\\
FileWriter out = new FileWriter(fileName);\\
int c;\\
while((c = in.read()) != -1)\{\\
\hspace{2mm} out.write(c);\\
\}\\}
Use ints to store last 16 bits instead of last 8 bits.\\

\subsection{Line-Oriented I/O}
\textit{BufferedReader.readLine()} uses line terminators to split the lines.\\
\textit{BufferedReader in = new BufferedReader(new FileReader(fileName));\\
PrintWriter out = new PrintWriter(new FileWriter(fileName));\\
String line;\\
while((line = in.readLine()) != null)\{\\
\hspace{2mm} out.println(line);\\
\}\\}

\subsection{Scanner}
\textit{Scanner s = new Scanner(<Stream>);\\
while(s.hasNext())\{\\
\hspace{2mm} doSomething...\\
\}}\\
Using different delimiter with: \textit{s.useDelimiter(Regex);}\\

\subsection{I/O From Command Line}
\textit{InputStreamReader in = new InputStreamReader(System.in);}\\
Or use \textbf{Console}:\\
\textit{Console c = System.console();}\\
Returns \textit{null} if not available.\\

\subsection{Data Streams}
Streams to read and write primitive data types.\\
\textit{DataInputStream in = new DataInputStream(new FileInputStream(fileName));\\
DataOutputStream out = new DataOutputStream(new BufferedOutputStream(new FileOutputStream(fileName)));\\
in.readDouble(); / out.writeDouble(someDouble);\\
in.readInt(); / out.writeInt(someInt);\\
in.readUTF(); / out.writeUTF(someString);\\
}

\subsection{Object Streams}
Java objects can be written to files if they implement the \textbf{Serializable} marker interface.\\
Every reference inside this object will also be written to the file.\\
Classes to use are \textbf{ObjectInputStream} and \textbf{ObjectOutputStream}.\\

\subsection{File I/O With NIO}
New since Java7.\\
Class \textbf{Path} to represent a path.\\
\textit{Path p = Paths.get("/tmp/foo");}\\
For releasing resources after use, use \textbf{try-with-resources}:\\
\textit{try(BufferedWriter writer = Files.newBufferedWriter(file, charset));}\\
For file handling use class \textbf{Files}.\\
\textit{Files.write(Path, byte[], OpenOption...);}\\
\textbf{OpenOptions}:\\
\{WRITE, APPEND, TRUNCATE\textunderscore EXISTING, CREATE\textunderscore NEW, CREATE, DELETE\textunderscore ON\textunderscore CLOSE, SPARSE, SYNC, DSYNC\}

\end{multicols}
\end{document}
