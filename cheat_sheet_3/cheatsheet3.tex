\documentclass[10pt,landscape]{article}
\usepackage{multicol}
\usepackage{calc}
\usepackage{ifthen}
\usepackage[landscape]{geometry}
\usepackage{hyperref}


% To Do:
% \listoffigures \listoftables
% \setcounter{secnumdepth}{0}


% This sets page margins to .5 inch if using letter paper, and to 1cm
% if using A4 paper. (This probably isn't strictly necessary.)
% If using another size paper, use default 1cm margins.
\ifthenelse{\lengthtest { \paperwidth = 11in}}
	{ \geometry{top=.5in,left=.5in,right=.5in,bottom=.5in} }
	{\ifthenelse{ \lengthtest{ \paperwidth = 297mm}}
		{\geometry{top=1cm,left=1cm,right=1cm,bottom=1cm} }
		{\geometry{top=1cm,left=1cm,right=1cm,bottom=1cm} }
	}

% Turn off header and footer
\pagestyle{empty}
 

% Redefine section commands to use less space
\makeatletter
\renewcommand{\section}{\@startsection{section}{1}{0mm}%
                                {-1ex plus -.5ex minus -.2ex}%
                                {0.5ex plus .2ex}%x
                                {\normalfont\large\bfseries}}
\renewcommand{\subsection}{\@startsection{subsection}{2}{0mm}%
                                {-1explus -.5ex minus -.2ex}%
                                {0.5ex plus .2ex}%
                                {\normalfont\normalsize\bfseries}}
\renewcommand{\subsubsection}{\@startsection{subsubsection}{3}{0mm}%
                                {-1ex plus -.5ex minus -.2ex}%
                                {1ex plus .2ex}%
                                {\normalfont\small\bfseries}}
\makeatother

% Define BibTeX command
\def\BibTeX{{\rm B\kern-.05em{\sc i\kern-.025em b}\kern-.08em
    T\kern-.1667em\lower.7ex\hbox{E}\kern-.125emX}}

% Don't print section numbers
\setcounter{secnumdepth}{0}


\setlength{\parindent}{0pt}
\setlength{\parskip}{0pt plus 0.5ex}


% -----------------------------------------------------------------------

\begin{document}

\raggedright
\footnotesize
\begin{multicols}{3}


% multicol parameters
% These lengths are set only within the two main columns
%\setlength{\columnseprule}{0.25pt}
\setlength{\premulticols}{1pt}
\setlength{\postmulticols}{1pt}
\setlength{\multicolsep}{1pt}
\setlength{\columnsep}{2pt}

\begin{center}
     \Large{\textbf{JAVA Cheat Sheet 3}} \\
\end{center}

\section{Basic OO}

\subsection{Reference types}
In JAVA only primitive types are handled \textbf{by value}, because their representation does have a fixed size.\\
All other objects are handled \textbf{by reference}.\\
Thus \textbf{only} their \textbf{references are passed}, not their values $\rightarrow$ Operations on variables storing references will affect other variables holding the same reference.

\subsection{Objects, classes, instances}
Every JAVA program is a set of \textbf{classes}.\\
Every piece of code has to be part of a class.\\
A class is the \textbf{concept} or \textbf{blueprint} of an object.\\
Classes can be \textbf{instantiated}, creating a concrete object from those blueprints - an \textbf{instance}.\\
Every class is inherited of the class \textbf{Object}.

\subsection{Attributes (fields, members)}
A class can have several \textbf{attributes} describing it. Those attributes are typically declared at the top of the class and their scope is the whole class. They are called \textbf{fields} or \textbf{members}.

\section{Advanced OO}

\subsection{Modifiers (class, attribute, method)}
\subsubsection{Visibility}
\begin{itemize}
\item public $\rightarrow$ Accessible from everywhere
\item private $\rightarrow$ Accessible only from within this class
\item protected $\rightarrow$ Accessible from this class, all sub classes and all classes in the same package.
\item package private (no keyword) $\rightarrow$ Accessible from this class and all classes in the same package
\end{itemize}
\subsubsection{static}
Can be accessed without having an instance of the class.\\
Holds for every instance.\\
\subsubsection{final}
Final \textbf{classes} may \textbf{not be subclassed}.\\
Final \textbf{methods} \textbf{cannot be overridden} by subclasses.\\
Final \textbf{attributes} \textbf{cannot change their values} once initialized.

\subsection{Constructors, initializers, memory allocation}

\subsection{Destructors, garbage collection}

\subsection{Inheritance}

\subsection{Polymorphism}

\subsection{Local, inner, anonymous classes}

\subsection{Modularization, packages, archives}







\end{multicols}
\end{document}
