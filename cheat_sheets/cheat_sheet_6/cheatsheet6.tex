\documentclass[10pt,landscape]{article}
\usepackage{multicol}
\usepackage{calc}
\usepackage{ifthen}
\usepackage[landscape]{geometry}
\usepackage{hyperref}
\usepackage{cite}
\usepackage{listings}


% To Do:
% \listoffigures \listoftables
% \setcounter{secnumdepth}{0}


% This sets page margins to .5 inch if using letter paper, and to 1cm
% if using A4 paper. (This probably isn't strictly necessary.)
% If using another size paper, use default 1cm margins.
\ifthenelse{\lengthtest { \paperwidth = 11in}}
	{ \geometry{top=.5in,left=.5in,right=.5in,bottom=.5in} }
	{\ifthenelse{ \lengthtest{ \paperwidth = 297mm}}
		{\geometry{top=1cm,left=1cm,right=1cm,bottom=1cm} }
		{\geometry{top=1cm,left=1cm,right=1cm,bottom=1cm} }
	}

% Turn off header and footer
\pagestyle{empty}
 

% Redefine section commands to use less space
\makeatletter
\renewcommand{\section}{\@startsection{section}{1}{0mm}%
                                {-1ex plus -.5ex minus -.2ex}%
                                {0.5ex plus .2ex}%x
                                {\normalfont\large\bfseries}}
\renewcommand{\subsection}{\@startsection{subsection}{2}{0mm}%
                                {-1explus -.5ex minus -.2ex}%
                                {0.5ex plus .2ex}%
                                {\normalfont\normalsize\bfseries}}
\renewcommand{\subsubsection}{\@startsection{subsubsection}{3}{0mm}%
                                {-1ex plus -.5ex minus -.2ex}%
                                {1ex plus .2ex}%
                                {\normalfont\small\bfseries}}
\makeatother

% Define BibTeX command
\def\BibTeX{{\rm B\kern-.05em{\sc i\kern-.025em b}\kern-.08em
    T\kern-.1667em\lower.7ex\hbox{E}\kern-.125emX}}

% Don't print section numbers
\setcounter{secnumdepth}{0}


\setlength{\parindent}{0pt}
\setlength{\parskip}{0pt plus 0.5ex}


% -----------------------------------------------------------------------

\begin{document}

\raggedright
\footnotesize
\begin{multicols}{3}


% multicol parameters
% These lengths are set only within the two main columns
%\setlength{\columnseprule}{0.25pt}
\setlength{\premulticols}{1pt}
\setlength{\postmulticols}{1pt}
\setlength{\multicolsep}{1pt}
\setlength{\columnsep}{2pt}

\begin{center}
     \Large{\textbf{JAVA Cheat Sheet 6\\I/O and GUIs}} \\
\end{center}

\section{I/O}
"A \textbf{stream} is a sequence if data."\\
"A program uses an \textbf{input stream} to read data from a source."\\
"A program uses an \textbf{output stream} to write data to a destination."  \textit{(docs.oracle.com)}\\
\subsection{Byte Streams}
In Java all stream types are built upon \textbf{ByteStreams}.
\begin{lstlisting}
FileInputStream in 
  = new FileInputStream(fileName);
FileOutputStream out 
  = new FileOutputStream(fileName);
int c;
while((c = in.read()) != -1){
  out.write(c);
}
\end{lstlisting}
\subsection{Character Streams}
\textbf{FileReader} and \textbf{FileWriter} use \textit{FileInputStream} and \textit{FileOutputStream} internally.\\
\begin{lstlisting}
FileReader in = new FileReader(fileName);
FileWriter out = new FileWriter(fileName);
int c;
while((c = in.read()) != -1){
 out.write(c);
}
\end{lstlisting}
Use ints to store last 16 bits instead of last 8 bits.\\

\subsection{Line-Oriented I/O}
\textit{BufferedReader.readLine()} uses line terminators to split the lines.\\
\begin{lstlisting}
BufferedReader in = 
  new BufferedReader(new FileReader(fileName));
PrintWriter out = 
  new PrintWriter(new FileWriter(fileName));
String line;
while((line = in.readLine()) != null){
 out.println(line);
}
\end{lstlisting}

\subsection{Scanner}
\begin{lstlisting}
Scanner s = new Scanner(<Stream>);
while(s.hasNext()){
 doSomething...
}
\end{lstlisting}
Using different delimiter with: 
\begin{lstlisting}
s.useDelimiter(Regex);
\end{lstlisting}

\subsection{I/O From Command Line}
\begin{lstlisting}
InputStreamReader in = 
  new InputStreamReader(System.in);
\end{lstlisting}
Or use \textbf{Console}:\\
\begin{lstlisting}
Console c = System.console();
\end{lstlisting}
Returns \textit{null} if not available.\\

\subsection{Data Streams}
Streams to read and write primitive data types.\\
\begin{lstlisting}
DataInputStream in = 
  new DataInputStream(
    new FileInputStream(fileName));
    
DataOutputStream out = 
  new DataOutputStream(
    new BufferedOutputStream(
      new FileOutputStream(fileName)));
      
in.readDouble(); / out.writeDouble(someDouble);
in.readInt(); / out.writeInt(someInt);
in.readUTF(); / out.writeUTF(someString);
\end{lstlisting}

\subsection{Object Streams}
Java objects can be written to files if they implement the \textbf{Serializable} marker interface.\\
Every reference inside this object will also be written to the file.\\
Classes to use are \textbf{ObjectInputStream} and \textbf{ObjectOutputStream}.\\

\subsection{File I/O With NIO}
New since Java7.\\
Class \textbf{Path} to represent a path.\\
\begin{lstlisting}
Path p = Paths.get("/tmp/foo");
\end{lstlisting}
For releasing resources after use, use \textbf{try-with-resources}:\\
\begin{lstlisting}
try(BufferedWriter writer = 
  Files.newBufferedWriter(file, charset));
\end{lstlisting}
For file handling use class \textbf{Files}.\\
\begin{lstlisting}
Files.write(Path, byte[], OpenOption...);
\end{lstlisting}
\textbf{OpenOptions}:\\
\{WRITE, APPEND, TRUNCATE\textunderscore EXISTING, CREATE\textunderscore NEW, CREATE, DELETE\textunderscore ON\textunderscore CLOSE, SPARSE, SYNC, DSYNC\}

\section{GUIs}
GUIs are event driven. There are \textbf{components} which can trigger \textbf{events}, and there are \textbf{event handlers} that can handle those events.
\subsection{Swing Components}
There are several swing components.
\begin{itemize}
\item JLabel
\item JTextField
\item JButton
\item JCheckBox
\item JComboBox
\item JList
\item JPanel
\end{itemize}

Those components can be added to container objects like JFrame or JPanel.
\begin{lstlisting}
JFrame frame = new JFrame();
frame.add(new JLabel());
frame.add(new JButton());
\end{lstlisting}

A \textbf{layout manager} determines where components are placed within a container.\\
To make a container or a component visible, the \textit{setVisible()} method has to be called.
\begin{lstlisting}
JFrame frame = new JFrame();
frame.setVisible(true);
\end{lstlisting}

\subsection{Event Handling}
Different components can create different events. Each event has to be handled by a specific event handler. The event handler has to register to the component in order to handle its events.\\
There are the following events:
\begin{itemize}
  \item EventObject
  \begin{itemize}
    \item EventObject
    \begin{itemize}
      \item AwtEvent
      \begin{itemize}
		\item ActionEvent
		\item AdjustmentEvent
		\item ItemEvent
		\item TextEvent
		\item ComponentEvent
	  \end{itemize}
    \end{itemize}
  \end{itemize}
\end{itemize}

\textit{continued...}\\
$\rightarrow$ComponentEvent
\begin{itemize}
  \item ContainerEvent
  \item FocusEvent
  \item PaintEvent
  \item WindowEvent
  \item InputEvent
  \begin{itemize}
	\item KeyEvent
	\item MouseEvent
	\begin{itemize}
	  \item MouseWheelEvent
	\end{itemize}
  \end{itemize}
\end{itemize}

There are also a lot of listeners which all extend EventListener:
\begin{itemize}
\item ActionListener
\item AdjustmentListener
\item ComponentListener
\item ContainerListener
\item FocusListener
\item ItemListener
\item KeyListener
\item MouseListener
\item MouseMotionListener
\item TextListener
\item WindowListener
\end{itemize}

To become an event handler a class has to implement the according event listener interface.
\begin{lstlisting}
class ButtonHandler implements ActionListener{
  public void actionPerformed(ActionEvent e){
    System.out.println("Button has been pressed");
  }
}
\end{lstlisting}

This handler then has to be added to the component.
\begin{lstlisting}
JButton button = new JButton();
button.addActionListener(new ButtonHandler());
\end{lstlisting}

\end{multicols}
\end{document}


