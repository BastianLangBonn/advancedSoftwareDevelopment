\documentclass[10pt,landscape]{article}
\usepackage{multicol}
\usepackage{calc}
\usepackage{ifthen}
\usepackage[landscape]{geometry}
\usepackage{hyperref}
\usepackage{cite}


% To Do:
% \listoffigures \listoftables
% \setcounter{secnumdepth}{0}


% This sets page margins to .5 inch if using letter paper, and to 1cm
% if using A4 paper. (This probably isn't strictly necessary.)
% If using another size paper, use default 1cm margins.
\ifthenelse{\lengthtest { \paperwidth = 11in}}
	{ \geometry{top=.5in,left=.5in,right=.5in,bottom=.5in} }
	{\ifthenelse{ \lengthtest{ \paperwidth = 297mm}}
		{\geometry{top=1cm,left=1cm,right=1cm,bottom=1cm} }
		{\geometry{top=1cm,left=1cm,right=1cm,bottom=1cm} }
	}

% Turn off header and footer
\pagestyle{empty}
 

% Redefine section commands to use less space
\makeatletter
\renewcommand{\section}{\@startsection{section}{1}{0mm}%
                                {-1ex plus -.5ex minus -.2ex}%
                                {0.5ex plus .2ex}%x
                                {\normalfont\large\bfseries}}
\renewcommand{\subsection}{\@startsection{subsection}{2}{0mm}%
                                {-1explus -.5ex minus -.2ex}%
                                {0.5ex plus .2ex}%
                                {\normalfont\normalsize\bfseries}}
\renewcommand{\subsubsection}{\@startsection{subsubsection}{3}{0mm}%
                                {-1ex plus -.5ex minus -.2ex}%
                                {1ex plus .2ex}%
                                {\normalfont\small\bfseries}}
\makeatother

% Define BibTeX command
\def\BibTeX{{\rm B\kern-.05em{\sc i\kern-.025em b}\kern-.08em
    T\kern-.1667em\lower.7ex\hbox{E}\kern-.125emX}}

% Don't print section numbers
\setcounter{secnumdepth}{0}


\setlength{\parindent}{0pt}
\setlength{\parskip}{0pt plus 0.5ex}


% -----------------------------------------------------------------------

\begin{document}

\raggedright
\footnotesize
\begin{multicols}{3}


% multicol parameters
% These lengths are set only within the two main columns
%\setlength{\columnseprule}{0.25pt}
\setlength{\premulticols}{1pt}
\setlength{\postmulticols}{1pt}
\setlength{\multicolsep}{1pt}
\setlength{\columnsep}{2pt}

\begin{center}
     \Large{\textbf{JAVA Cheat Sheet 3}} \\
\end{center}

\section{Basic OO}

\subsection{Reference types}
In JAVA only primitive types are handled \textbf{by value}, because their representation does have a fixed size.\\
All other objects are handled \textbf{by reference}.\\
Thus \textbf{only} their \textbf{references are passed}, not their values $\rightarrow$ Operations on variables storing references will affect other variables holding the same reference.

\subsection{Objects, classes, instances}
Every JAVA program is a set of \textbf{classes}.\\
Every piece of code has to be part of a class.\\
A class is the \textbf{concept} or \textbf{blueprint} of an object.\\
Classes can be \textbf{instantiated}, creating a concrete object from those blueprints - an \textbf{instance}.\\
Every class is inherited of the class \textbf{Object}.

\subsection{Attributes (fields, members)}
A class can have several \textbf{attributes} describing it. Those attributes are typically declared at the top of the class and their scope is the whole class. They are called \textbf{fields} or \textbf{members}.

\section{Advanced OO}

\subsection{Modifiers (class, attribute, method)}
\subsubsection{Visibility}
\begin{itemize}
\item \textbf{public} $\rightarrow$ Accessible from everywhere
\item \textbf{private} $\rightarrow$ Accessible only from within this class
\item \textbf{protected} $\rightarrow$ Accessible from this class, all sub classes and all classes in the same package.
\item package private (no keyword) $\rightarrow$ Accessible from this class and all classes in the same package
\end{itemize}
\subsubsection{static}
Can be accessed without having an instance of the class.\\
Holds for every instance.\\
\subsubsection{final}
Final \textbf{classes} may \textbf{not be subclassed}.\\
Final \textbf{methods} \textbf{cannot be overridden} by subclasses.\\
Final \textbf{attributes} \textbf{cannot change their values} once initialized.

\subsection{Constructors, initializers, memory allocation}
\subsubsection{Constructors}
Methods called when \textbf{initializing} an Object.\\
Do not specify a return value.\\
Name has to equal the class' name.\\
Every object provides a \textbf{default constructor}.\\
Usually used to initialize fields with user given values.\\

\subsubsection{Initializers}
Code that \textbf{does not belong to any method} inside of a class.\\
Declaring (and initializing) fields is a common usage.\\
Will be executed in order of appearance.
\footnote{see http://www.dummies.com/how-to/content/what-is-an-initializer-in-java.html}\\

\subsubsection{Memory Allocation}
JAVA objects reside in the \textbf{heap}.\\
The \emph{heap} is divided into \textbf{young space} and \textbf{old space}.\\
The \textbf{JVM} distinguishes between \textbf{small} and \textbf{large objects}.\\
\emph{Small objects} get allocated in \textbf{thread local areas (TLAs)} of the \emph{heap}.\\
\emph{Large objects} are allocated in the \emph{old space}.\footnote{see https://docs.oracle.com/cd/E13150\textunderscore 01/jrockit\textunderscore jvm/jrockit/geninfo/diagnos/garbage\textunderscore collect.html}\\

\subsection{Destructors, garbage collection}

\subsubsection{Destructors}
In JAVA there are no destructors. Destroying objects is \textbf{done by the garbage collector}, but there is no exact telling when this will be done.

\subsubsection{Garbage Collection}
\footnote{see https://docs.oracle.com/cd/E13150\textunderscore 01/jrockit\textunderscore jvm/jrockit/geninfo/diagnos/garbage\textunderscore collect.html}
\textbf{Garbage collection} is the process of freeing space in the \emph{young space} for the allocation of new objects.\\
The \emph{JVM} uses \textbf{mark and sweep} for garbage collection on the \textbf{whole heap}.\\
It \textbf{marks} all objects as \emph{alive} that are \"reachable\" from threads, native handles and other root sources and also those objects reachable from \emph{alive} objects. The rest is considered garbage.\\
It then identifies the free space between the \emph{alive} objects. This is called \textbf{sweeping}.\\
Garbage collection on the \textbf{young space} is called \textbf{young collection}.\\
The \textbf{young collector} promotes all live objects outside a so called \textbf{keep area} to the \textbf{old space}.\\
A garbage collection strategy using a \emph{young space} is called \textbf{generational garbage collection strategy}.\\
There are several \textbf{garbage collection modes}:\\
\textbf{dynamic modes:}
\begin{itemize}
\item throughput $\rightarrow$ maximum application throughput
\item pausetime $\rightarrow$ short and even pause times
\item determinicstic $\rightarrow$ very short and deterministic pause times
\end{itemize}
\textbf{static modes:}
\begin{itemize}
\item singlepar $\rightarrow$ single-generational parallel garbage collector
\item genpar $\rightarrow$ two-generational parallel garbage collector
\item singlecon $\rightarrow$ single-generational mostly concurrent garbage collector
\item gencon $\rightarrow$ two-generational mostly concurrent garbage collector
\end{itemize}
\textbf{Compaction} is used to \textbf{get rid of small chunks of space} in the heap and to instead \textbf{create bigger chunks} by moving objects in the heap closer together.

\subsection{Inheritance}
In JAVA except for the class \emph{Object} every class has exactly one \textbf{superclass}.\\
A \textbf{subclass} inherits fields and methods from its superclass.\\
Eventually, every class inherits from the class \emph{Object}.\\
To subclass another class use keyword \textbf{extends}.\\
\emph{public class Audi \textbf{extends} Car\{...\}}

\subsection{Polymorphism}
In JAVA every class is also an instance of its superclass.\\
As for this holds also for the superclass, every class in JAVA may be an instance of many different classes.\\
Eventually every class is an instance of \emph{Object}.\\
\textbf{possible:}\\
\emph{Car car = new Audi();} $\leftarrow$ if Audi extends Car.\\
But now only functionality of type Car is accessible, even if the assigned type is an Audi. 

\subsection{Local, inner, anonymous classes}
An \textbf{inner class} is a class defined within another class. So the inner class can only be accessed from an instance of the outer class.\\
A \textbf{locale class} is a class that is defined in a block of another class, typically a method.\\
A local class can only be used within the block where it is declared - it has \textbf{local scope}.\\
\emph{OuterClass.InnerClass innerObject = outerObject.new InnerClass();}
\footnote{see https://docs.oracle.com/javase/tutorial/java/javaOO/nested.html}\\
An \textbf{anonymous class} is a class that is declared and instantiated at the same time.\\
\emph{Car car = new Audi()\{...$<$class definition here$>$...\}}

\subsection{Modularization, packages, archives}
\textbf{Packages} and \textbf{archives} are ways to \textbf{modularize code}. Packages have a hierarchical \textbf{tree-like structure} and function as folders for code.\\
\textbf{JAVA archive files (JARs)} are archives containing packages and their contained classes.


\end{multicols}
\end{document}
