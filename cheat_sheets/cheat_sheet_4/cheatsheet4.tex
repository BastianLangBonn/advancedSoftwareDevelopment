\documentclass[10pt,landscape]{article}
\usepackage{multicol}
\usepackage{calc}
\usepackage{ifthen}
\usepackage[landscape]{geometry}
\usepackage{hyperref}
\usepackage{cite}
\usepackage{listings}
\usepackage{graphicx}
\usepackage{float}

% To Do:
% \listoffigures \listoftables
% \setcounter{secnumdepth}{0}


% This sets page margins to .5 inch if using letter paper, and to 1cm
% if using A4 paper. (This probably isn't strictly necessary.)
% If using another size paper, use default 1cm margins.
\ifthenelse{\lengthtest { \paperwidth = 11in}}
	{ \geometry{top=.5in,left=.5in,right=.5in,bottom=.5in} }
	{\ifthenelse{ \lengthtest{ \paperwidth = 297mm}}
		{\geometry{top=1cm,left=1cm,right=1cm,bottom=1cm} }
		{\geometry{top=1cm,left=1cm,right=1cm,bottom=1cm} }
	}

% Turn off header and footer
\pagestyle{empty}
 

% Redefine section commands to use less space
\makeatletter
\renewcommand{\section}{\@startsection{section}{1}{0mm}%
                                {-1ex plus -.5ex minus -.2ex}%
                                {0.5ex plus .2ex}%x
                                {\normalfont\large\bfseries}}
\renewcommand{\subsection}{\@startsection{subsection}{2}{0mm}%
                                {-1explus -.5ex minus -.2ex}%
                                {0.5ex plus .2ex}%
                                {\normalfont\normalsize\bfseries}}
\renewcommand{\subsubsection}{\@startsection{subsubsection}{3}{0mm}%
                                {-1ex plus -.5ex minus -.2ex}%
                                {1ex plus .2ex}%
                                {\normalfont\small\bfseries}}
\makeatother

% Define BibTeX command
\def\BibTeX{{\rm B\kern-.05em{\sc i\kern-.025em b}\kern-.08em
    T\kern-.1667em\lower.7ex\hbox{E}\kern-.125emX}}

% Don't print section numbers
\setcounter{secnumdepth}{0}


\setlength{\parindent}{0pt}
\setlength{\parskip}{0pt plus 0.5ex}


% -----------------------------------------------------------------------

\begin{document}

\raggedright
\footnotesize
\begin{multicols}{3}


% multicol parameters
% These lengths are set only within the two main columns
%\setlength{\columnseprule}{0.25pt}
\setlength{\premulticols}{1pt}
\setlength{\postmulticols}{1pt}
\setlength{\multicolsep}{1pt}
\setlength{\columnsep}{2pt}

\begin{center}
     \Large{\textbf{JAVA Cheat Sheet 4\\Data Structures and Advanced Concepts}} \\
\end{center}
\textit{If not noted otherwise most of the information is taken from Kathy Sierra, Bert Bates, "Sun Certified Programmer for Java 6", The McGraw-Hill Companies, 2008}
\section{Data Structures}
\subsection{Arrays}

\begin{itemize}
\item Objects that store multiple variables of the same type.
\item Can hold either primitives or object references.
\item Will always be an object on the heap.
\end{itemize}
\subsubsection{Declaration}
\begin{lstlisting}
// common
int[] key; 
// better avoid due to readability
int key [];
\end{lstlisting}
\subsubsection{Construction}
Creates the object on the heap.\\

\begin{lstlisting}
// The size has to be specified
int[] myArray = new int[3];
\end{lstlisting}

For multidimensional arrays omitting the second dimension's size is possible:\\

\begin{lstlisting}
int[][] myArray = new int[3][];
\end{lstlisting}
\subsubsection{Initializing}
Arrays can be initialized by assigning values to their elements:\\
\begin{lstlisting}
int[] myArray = int[3];
myArray[0] = 5;
\end{lstlisting}
The index always starts with 0 and the highest index equals the array's length -1.\\
Declaration, construction and initialization is also possible in one single step:\\
\begin{lstlisting} 
int[] dots = {1,3,5}
\end{lstlisting}

\subsubsection{Anonymous Array Creation}
It is possible to create an anonymous array, for example to pass it as a functions parameter.\\
\begin{lstlisting}
functionCall(new int[]{1,2,3});
\end{lstlisting}

\subsection{Enumerations}
The JAVA type for enumerations is called enum.\\
The restrict the values for a variable to a predefined set of values.\\
\begin{lstlisting}
enum CoffeeSize{
  BIG, HUGE, OVERWHELMING
};
\end{lstlisting}
The ';' at the end is optional.\\
An enum can be seen as a special kind of class. The values are constants (public static final) instances of this class. One can define constructors and other methods:
\begin{lstlisting}
enum CoffeeSize{
  BIG(8), HUGE(10), OVERWHELMING(16);
  CoffeeSize(int ounces){
    this.ounces = ounces;
  }
  private int ounces;
  private int getOunces(){
    return this.ounces;
  }
}
\end{lstlisting} 
Enums also can have a constant specific class body which overrides a method:\\
\begin{lstlisting}
  ...
  OVERWHELMING(16){
    public String getLidCode(){
      return "B";
    }
  };
  ...
\end{lstlisting}

The enum-class implements equals and hashCode, therefore it is possible to use enums as keys for maps.

\subsection{Lists, Trees, Collections}
In JAVA we distinguish between \textbf{collections}, a \textbf{Collection} and the utility class \textbf{Collections}.\\
There are many classes considered to fulfil the concept of a collection. Those are collections. Then there is the class Collection, which is the interface for a subset of these collections. And then there is a utility class providing static methods for collections.\\
The single collections cannot take primitives as elements. In this case one have to use the wrapper types. Java implicitly uses \textbf{auto-boxing}, so it is possible to give primitives as arguments instead of their wrapper type representations.
\subsubsection{Overview over the collections}
\begin{itemize}
	\item Collection
	\begin{itemize}
		\item Set
		\begin{itemize}
			\item HashSet
			\begin{itemize}
				\item LinkedHashSet
			\end{itemize}
			\item SortedSet
		\end{itemize}
		\item List
		\begin{itemize}
			\item ArrayList
			\item Vector
			\item LinkedList
		\end{itemize}
		\item Queue
		\begin{itemize}
			\item LinkedList
			\item PriorityQueue
		\end{itemize}
	\end{itemize}
	\item Map
	\begin{itemize}
		\item HashTable
		\item HashMap
		\begin{itemize}
			\item LinkedHashMap
		\end{itemize}
		\item SortedMap
		\begin{itemize}
			\item Navigable Map
			\begin{itemize}
				\item TreeMap
			\end{itemize}
		\end{itemize}
	\end{itemize}
\end{itemize}

\subsubsection{Lists}
There are three implementations of the interface List. They all provide the common methods of Collection and additionally provide methods related to the index of an element.\\
\textbf{ArrayList}\\
An ArrayList provides fast access and fast iterations over its elements.\\
But every time its allocated memory is exceeded by inserting new elements a new list has to be created and the whole old list has to copied into the new list.\\
\begin{lstlisting}
List<String> list = new ArrayList<String>();
list.add("hello");
String string = list.get(0);
\end{lstlisting}
\textbf{Vector}\\
Leftover from earlier JAVA versions. Mainly the same as an ArrayList, but uses synchronized methods for the use in multi-threading, which makes it slower.\\
\begin{lstlisting}
List<String> list = new Vector<String>();
list.add("hello");
String string = list.get(0);
\end{lstlisting}
\textbf{LinkedList}\\
A double-linked list. Provides methods for adding from the beginning or the end, which makes it a good choice for queues or stacks. Insertion and Deletion is faster than for the other two lists, but iteration and access may be slower.
\begin{lstlisting}
List<String> list = new LinkedList<String>();
list.add("hello");
String string = list.get(0);
\end{lstlisting}
\subsubsection{Trees}
There are only two structures using trees:
\begin{itemize}
\item TreeSet
\item TreeMap
\end{itemize}
Both provide an ascending order for the elements based on the elements' natural order.\\
The methods \textbf{lower}($<$), \textbf{floor}($\leq$) and \textbf{higher}($>$), \textbf{ceiling}($\geq$) can be used to navigate or search within tree structures.\\

\subsubsection{Collections}
This class contains utility methods for collections, for example to search elements, sort collections, reverse the order of a list and so on.

\section{Advanced Concepts}
\subsection{Iterators}
An \textbf{iterator} is an object that lets one loop over a collection step by step.\\
\begin{lstlisting}
Iterator<Dog> iterator = d.iterator();
while(iterator.hasNext()){
  Dog dog = iterator.next();
  System.out.println(dog.name);
}
\end{lstlisting}

\subsection{Recursion}
If a function calls itself, it is called recursive.
\begin{lstlisting}
int factorial(int k)
{
  if(k == 1)
    return 1;
  else
  {
    return(k*(factorial(k-1)));
  }
} 
\end{lstlisting}

\subsection{Templates}
There is no concept called 'templates' in JAVA. For generic programming JAVA uses generics.

\subsection{Generics}
Generics are a way to ensure type safety for classes that uses variable types. Collections are the most common example. The angle brackets define the type of the elements used in the collection and ensure that the taken and returned objects are of the specific type. This way the programmer does not need to cast and may run into \textit{ClassCastExceptions}.\\
Note that it is not possible to use sub- or super-types as generic types. For the base type this is okay.
\begin{lstlisting}
// valid
List<Integer> myList = new ArrayList<Integer>();
// not valid
ArrayList<Animal> myList = new ArrayList<Dog>();
\end{lstlisting}

To be able to use also sub/super types, one can use the wildcard '?':
\begin{lstlisting}
public void addAnimal(List<? super Dog> animals){
  animals.add(new Dog());
}
\end{lstlisting}


\subsection{Reflection}
Reflection provides the possibility to have a look at the functions an object provides during runtime.
\begin{lstlisting}
Method method = myClass.getClass()
  .getMethod("doSomething", null);
method.invoke(myClass, null);
\end{lstlisting}

\subsection{Annotation}
"Annotations, a form of metadata, provide data about a program that is not part of the program itself. Annotations have no direct effect on the operation of the code they annotate."\\
(https://docs.oracle.com/javase/tutorial/java/annotations/)\\
There are seven predefined annotations:
\begin{itemize}
\item @Deprecated
\item @Override
\item @SuppressWarnings
\item @SafeVarargs
\item @Documented
\item @Inherited
\item @Retention
\item @Target
\end{itemize}
The user can also create his own annotations.
\begin{lstlisting}
@interface Person {
  String name();
  int alter();
}
\end{lstlisting}

And use it:
\begin{lstlisting}
@Person(name = "Andreas Solymosi", alter = 56)
Konto konto = new Konto();
\end{lstlisting}
(https://de.wikipedia.org/wiki/Annotation\textunderscore\%28Java\%29)

\end{multicols}
\end{document}
